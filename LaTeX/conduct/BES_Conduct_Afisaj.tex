\documentclass[epjST]{svjour}
\usepackage{graphics}
\usepackage{ntheorem}
\usepackage{hyperref}
\usepackage{textcomp}
\hypersetup{
  colorlinks=true,
  linkcolor=cyan,
  filecolor=cyan,
  urlcolor=cyan,
  pdfpagemode=FullScreen,
}

\begin{document}
  \institute{} \headnote {Departamentul Media}
  \title{Angajamentul echipei de afișaj al bisericii Emanuel Sibiu}
  \author{Ioan Lucut\fnmsep
  \thanks{\email{ioan.lucut88@gmail.com}}}
  \abstract{Departamentul de afișaj al bisericii Emanuel din Sibiu este vital
  pentru programele bisericii, pentru trupele de laudă și închinare și pentru întreaga
  comunitate. Acest departament constă în organizarea afișării corecte d.p.d.v. gramatical
  al conținutului de cântece sau materiale relevante pentru întreaga biserică,
  fie în ProPresenter, fie în PowerPoint sau alte modalități precum și afișarea versetelor
  biblice în timpul predicilor sau îndemnurilor.}

  \maketitle
  {\textbf{Biserica Emanuel Sibiu} are nevoie de o organizare bună al acestui departament iar acest document vine în ajutorul celor care vor sa se implice, cu scopul de a trasa obiective clare și reguli pe care trebuie să le păstrăm și la care ne angajăm atunci când ne alăturăm acestei slujiri.}
  %

  \section{Standardele dorite}
  \label{sec:1} E nevoie să avem o abordare adecvată, precum:
  \begin{itemize}
    \item Trebuie să respectăm deciziile luate de cei responsabili (Ionel Lucuț
      și Laurențiu Văcar)

    \item Trebuie să avem disponibilitatea de a colabora în echipă

    \item Trebuie să fim consecvenți în slujire

    \item Trebuie să avem disponibilitatea și dorința de a progresa și a ne pregăti
      într-un mod continuu (Ionel Lucuț se va ocupa de instruire)

    \item Un prim curs al modului în care folosim aplicația ProPresenter poate fi
      vizualizat \href{https://t.me/bes_team/48/302}{aici}.
  \end{itemize}
  %

  \section{Reguli generale}
  \label{sec:2}
  \begin{itemize}
    \item Avem un scop comun de a afișa doar versiunile corecte d.p.d.v.
      gramatical, iar pentru asta vom defini o echipă care va aproba și centraliza
      versurile, numită echipa \textbf{Versuri} formată din Ionel Lucuț, Tania
      Lucuț, Diana Manta.

    \item Computerul va fi lăsat mereu pornit pentru a putea avea acces la ProPresenter
      via TeamViewer.

    \item Toate datele pe care le folosim vor fi salvate în Google Drive.

    \item Înainte să adăugăm resurse noi, ne vom asigura că nu există în alt loc
      în sistemul nostru, pentru nu a crea duplicate.

    \item Suntem repartizați în slujirea aceasta astfel încât să nu existe program
      de biserică fără persoane instruite să se ocupe de afișaj.

    \item \textbf{Nu} mai afișăm cântece în format PowerPoint.

    \item Prezentările PowerPoint existente ale fraților din biserică vor fi convertite
      în ProPresenter.

    \item Înaintea predicilor, dacă există prezentări PowerPoint ale fraților
      din biserică, vor fi comunicate în timp util echipei pentru a fi convertite
      în ProPresenter.
  \end{itemize}
  %

  \section{Echipa Versuri}
  \label{sec:4} Ce vom face:
  \begin{itemize}
    \item Versurile sunt centralizate într-un \textbf{singur loc} într-un
      repository în \href{https://github.com/ioanlucut/bes-lyrics}{Github}, având
      posibilitatea de a verifica amănunțit fiecare schimbare a cântecelor și de
      a genera din același loc caietul de cântece in PDF și acorduri.

    \item Acceptăm doar dacă este imperios necesar versiuni diferite ale cântecelor.

    \item Oricine poate să propună cântece și să le redacteze; doar membrii
      echipei \textbf{Versuri} vor putea să corecteze și verifice cântecele propuse,
      pentru a deveni oficiale în arhiva noastră.

    \item Momentan, cântecele sunt generate pentru formatul așteptat de
      ProPresenter (.pro), iar ele vor fi prezente mereu în Google Drive și
      accesibile de pe computerul de la biserică. Echipa \textbf{Versuri} se va
      ocupa ca ele să fie mereu prezente în program.
  \end{itemize}
  %

  \section{Reguli de afișare a versurilor}
  \label{sec:5} E nevoie să stabilim niște reguli cu privire la afișarea
  cântecelor în biserică:
  \begin{itemize}
    \item Fiecare cântec afișat în biserică trebuie să fie centralizat și verificat
      de către echipa \textbf{Versuri}.

    \item Versurile afișate sunt doar cele corectate/verificate și aprobate echipa
      \textbf{Versuri}.

    \item În fiecare duminică vom păstra un log centralizat cu schimbările necesare
      care trebuie făcute la versuri de către cel/cea care este la computer. De
      exemplu dacă în timp ce cântam o cântare, observăm că un cuvânt sau o expresie
      nu sună bine și trebuie rescris(ă), vom scrie în acel log iar ulterior vom
      comunica echipei \textbf{Versuri} pentru a corecta cântecele respective. Acest
      log va fi prezent în Google Drive și poate fi accesat \href{https://docs.google.com/spreadsheets/d/1-YsjtGIwzpnJNoVPZ4vhv8DvmeUYrmwuhKL6a5BtzXc/edit?usp=drive_link}{aici}.
  \end{itemize}
  %

  \section{Duminica}
  \label{sec:3}
  \begin{itemize}
    \item Înainte cu o zi sau două de programe, ne vom asigura că piesele sunt
      scrise, corectate, în ProPresenter și sunt manual referențiate într-un
      playlist în ProPresenter.

    \item Face parte din atribuțiunile persoanei care se va ocupa de afișaj în timpul
      programului de duminică să pregătească playlistul în prealabil.

    \item \textbf{Duminica nu copiem cântece în timpul bisericii de pe resurse
      creștine sau din alte locuri pentru a le afișa decât sub subpravegherea
      cuiva din echipa \textbf{Versuri}.} Mai degrabă cântăm un cântec din
      memorie, decât unul cu potențiale greșeli gramaticale care ne dezonorează.

    \item Înainte cu 15 minute de începerea programului, vom afișa un colaj în
      ProPresenter și o muzică de fundal.

    \item După încetarea programului, vom afișa un colaj în ProPresenter și o
      muzică de fundal.
  \end{itemize}
  %

  \section{Marți seara}
  \label{sec:6} Marția nu transmitem live și există situația în care trebuie să afișăm
  cântece din harfă sau din anumite caiete, care nu există în programul nostru. Așa
  că:
  \begin{itemize}
    \item Putem să adăugăm ad-hoc cântece de pe resurse creștine de care avem nevoie
      pentru biserică.

    \item Aceste versuri vor fi scrise într-o librărie diferită în ProPresenter și
      ulterior comunicate echipei \textbf{Versuri} pentru a fi centralizate.
  \end{itemize}
  %

  \section{ProPresenter}
  \label{sec:7} Datorită faptului că vrem să avem același nivel de profesionalism,
  vrem să propunem următoarele:
  \begin{itemize}
    \item Folosim o singură temă pentru cântecele din ProPresenter.

    \item Tema/fonturile/setările folosite nu vor fi schimbate decât de cei responsabili.

    \item Folosim biblia din ProPresenter.

    \item Nu schimbăm setări ad-hoc în sistem, ci păstrăm mereu ordine astfel
      încât oricine ar urma după noi să poată să continue slujirea.

    \item Toți cei implicați vor urma cursurile pregătite de cei responsabili despre
      cum folosim aplicația.
  \end{itemize}
  %

  \section{Echipele de laudă și închinare}
  \label{sec:8}
  \begin{itemize}
    \item Vor transmite cu cel puțin 2 zile înainte lista finală de cântece propuse
      pentru duminică/joi.

    \item Cântecele noi adăugate de echipele de laudă și închinare vor fi comunicate
      din timp echipei \textbf{Versuri}.
  \end{itemize}
\end{document}