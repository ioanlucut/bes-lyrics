% =====================================================================
% This file is auto-generated.
%
% Copyright (c) 2024 Ioan Lucuț (ioan.lucut88@gmail.com)
%
% Do not edit this file directly. Any changes made will be overwritten
% the next time the file is generated.
% =====================================================================

% This is the only preamble definition required
\documentclass{leadsheet}
\begin{document}

  %  https://tex.stackexchange.com/questions/9852/what-is-the-difference-between-page-break-and-new-page
  %  Every song should start in a new page.
  \newpage

  \begin{song}{
     title={Tu domnești pe tronul slavei în veșnicie îmbrăcat},
     subtitle={v1.1,v1.2,c1.1,c1.2,v2.1,v2.2,c1.1,c1.2,b,c1.1,c1.2},
     composer={Sunny Tranca}
  }

  \begin{verse}
    ^{F}Tu domnești pe tronul ^{Fmaj}slavei \\
    În veșni^{Bb/F}cie îmbrăcat ^{F-Bb/F}, \\
    Strălu^{F}cirea slavei ^{Fmaj}Tale \\
    Mă ^{Bb/F}uimește ne-n^{F-C/E}cetat.
  \end{verse}

  \begin{verse}
    Cine ^{Dm}oare se com^{C}pară, \\
    Doamne, ^{Bb}cu puterea ^{F}Ta? \\
    Tu ești ^{Gm}Domnul vieții ^{F/A}mele, \\
    Tu ești ^{Bb}mântu^{C}irea ^{F}mea.
  \end{verse}

  \begin{chorus}
    /: ^{F}Laudă, cinste, ^{C/E}onoare, \\
    Mie^{Bb/D}lului ce-a învi^{F/C}at. \\
    Noi aducem închinare Nume^{Bb/D}lui glo^{C}rifi^{F}cat.
  \end{chorus}

  \begin{chorus}
    ^{F}Laudă, cinste, o^{C/E}noare, \\
    Mielu^{Bb/D}lui ce stă pe ^{F/C}tron. \\
    Noi a^{F}ducem închi^{C/E}nare, doar pe ^{Bb/D}El Îl î^{C}năl^{F}țăm. :/
  \end{chorus}

  \begin{verse}
    Îngerii și heruvimii \\
    Fredonează toți în cor: \\
    El e vrednic de mărire, \\
    Doar El e biruitor.
  \end{verse}

  \begin{verse}
    Peste moarte ai călcat, \\
    Doamne, cu puterea Ta. \\
    Tu ești Domnul vieții mele, \\
    Tu ești mântuirea mea.
  \end{verse}

  \begin{bridge}
    Peste ^{Dm}moarte ai căl^{C}cat, \\
    Doamne, ^{Bb}cu puterea Ta. \\
    Tu ești ^{Gm}Domnul vieții ^{F/A}mele, \\
    ^{F/A}Tu ești ^{Bb}mântu^{C}irea ^{F}mea.
  \end{bridge}

  \end{song}
\end{document}