% =====================================================================
% This file is auto-generated.
%
% Copyright (c) 2024 Ioan Lucuț (ioan.lucut88@gmail.com)
%
% Do not edit this file directly. Any changes made will be overwritten
% the next time the file is generated.
% =====================================================================

% This is the only preamble definition required
\documentclass{leadsheet}
\begin{document}

  %  https://tex.stackexchange.com/questions/9852/what-is-the-difference-between-page-break-and-new-page
  %  Every song should start in a new page.
  \newpage

  \begin{song}{
     title={Peste munte e-un râu ce curge},
     subtitle={v1,c,v2,c,v3,c},
     composer={Agape}
  }

  \begin{verse}
    ^{A}Peste ^{E}munte e-un ^{B}râu ce ^{C#m}curge, \\
    În^{A}viorare a^{B}duce cur^{E}gând. \\
    Prin ^{A}văi a^{E}dânci și în^{B}tinsa câm^{C#m}pie, \\
    ^{A}Râul stră^{E}bate, ^{B}viață-adu^{E}când.
  \end{verse}

  \begin{chorus}
    /: E râul sfânt care ^{B7}suflete ^{E}mișcă, \\
    E râul sfânt care ^{B7}curge șu^{E}voi, \\
    Redeșteptarea de ^{B7}la Ru^{E}salii \\
    Re^{A}vars-o a^{E}cum, din ^{B7}plin peste ^{E}noi! :/
  \end{chorus}

  \begin{verse}
    Curgerea lui e venită din ceruri \\
    Și poate lua generații la rând. \\
    Cei ce străbat ale râului țărmuri \\
    Se-adapă din el, și-apoi cântă strigând.
  \end{verse}

  \begin{verse}
    Tuturora spre munte ne place să mergem, \\
    Biserica-ntreagă trezire-așteptând, \\
    În apă intrăm, la-nceput pân’ la glezne \\
    Și-apoi înspre larg ‘naintăm cu avânt.
  \end{verse}

  \end{song}
\end{document}