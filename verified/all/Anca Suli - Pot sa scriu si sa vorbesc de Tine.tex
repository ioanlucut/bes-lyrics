% =====================================================================
% This file is auto-generated.
%
% Copyright (c) 2024 Ioan Lucuț (ioan.lucut88@gmail.com)
%
% Do not edit this file directly. Any changes made will be overwritten
% the next time the file is generated.
% =====================================================================

% This is the only preamble definition required
\documentclass{leadsheet}
\begin{document}

  %  https://tex.stackexchange.com/questions/9852/what-is-the-difference-between-page-break-and-new-page
  %  Every song should start in a new page.
  \newpage

  \begin{song}{
     title={Pot să scriu și să vorbesc de Tine},
     subtitle={v1.1,v1.2,c,v2.1,v2.2,c},
     composer={Anca Suli}
  }

  \begin{verse}
    {G}Pot să scriu și ^{D/F#}să vorbesc de ^{Em}Tine, \\
    Despre ^{C}tot ce Tu ai ^{Am}spus și ai fă^{D}cut, \\
    Cum cu ^{G}pâini și pești ai ^{D/F#}săturat mulți^{Em}mea \\
    Și în ^{C}dragoste pe ^{Am}cruce ne-ai năs^{D}cut.
  \end{verse}

  \begin{verse}
    Cum la ^{Em}toți cei orbi doar ^{Eb}Tu le-ai dat ve^{Bm}dere \\
    Și cum ^{C}marea-nvolburată-ai liniș^{G}tit, \\
    Când ^{Em}eram eu singur ^{Eb}și în deznă^{Bm}dejde, \\
    Tu cu ^{C}dragoste în ^{Am}brațe m-ai pri^{D}mit!
  \end{verse}

  \begin{chorus}
    Tu ^{G}ne-ai ^{Am/G}adus iu^{G}birea, \\
    Ca ^{G}pe-un veș^{Am/G}mânt cu^{C}rat, \\
    Dar ^{G}pentru a^{C/G}cestea ^{G}toate, \\
    I^{Am}sus, ^{G/B}noi ^{D}ce Ți-am ^{G}dat?
  \end{chorus}

  \begin{verse}
    Când umbra acestei lumi e peste tine \\
    Și-ntunericul e în inima ta, \\
    Când tu simți că barca vieții e fără cârmă, \\
    Plutind singură în noaptea grea.
  \end{verse}

  \begin{verse}
    Dar eu știu că-ntotdeauna ești cu mine, \\
    Dar eu Numele Tău trebuie să-L rostesc, \\
    Știu ca Tu mă vei iubi întotdeauna, \\
    Dacă voia Ta, Isuse, o-mplinesc.
  \end{verse}

  \end{song}
\end{document}