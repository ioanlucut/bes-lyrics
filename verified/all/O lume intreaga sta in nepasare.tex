% =====================================================================
% This file is auto-generated.
%
% Copyright (c) 2024 Ioan Lucuț (ioan.lucut88@gmail.com)
%
% Do not edit this file directly. Any changes made will be overwritten
% the next time the file is generated.
% =====================================================================

% This is the only preamble definition required
\documentclass{leadsheet}
\begin{document}

  %  https://tex.stackexchange.com/questions/9852/what-is-the-difference-between-page-break-and-new-page
  %  Every song should start in a new page.
  \newpage

  \begin{song}{
     title={O lume întreagă stă în nepăsare},
     subtitle={v1.1,v1.2,c,v2.1,v2.2,c,v3.1,v3.2,c}
  }

  \begin{verse}
    ^{B}O lume în^{E}treagă stă în nepă^{E7}sare \\
    Și caută adă^{A}post când vine o-ncer^{E}care, \\
    Dar eu nu mă tem, ^{C#m}dar eu nu mă ^{B}tem.
  \end{verse}

  \begin{verse}
    În ceasul cel ^{E}bun e stăpână pe ^{E7}toate, \\
    Când Domnul îi vor^{A}bește, i-e frică de ^{A}moarte \\
    Dar eu nu mă ^{C#m}tem, ^{B}dar eu nu mă ^{E}tem!
  \end{verse}

  \begin{chorus}
    ^{E}Dar eu nu mă tem, \\
    ^{E7}chiar dacă cei din {A}jur mă pără^{E}sesc. \\
    Dar eu nu mă tem, orice-ar ve^{F#-B}ni, \\
    Viața ^{E-E7}mea e-n mâna Celui ^{A}ce-a murit pe ^{E}lemn, \\
    Cu mine Îl ^{C#m}chem, ^{B}și eu nu mă ^{E}tem!
  \end{chorus}

  \begin{verse}
    Ades’ vine-ncercarea, năvalnic ca marea \\
    Și norii negri grabnic întunecă zarea, \\
    Dar eu nu mă tem, dar eu nu mă tem.
  \end{verse}

  \begin{verse}
    Se spulberă cuvântul și orice speranță \\
    Și nici măcar o clipă nu mai e siguranță, \\
    Dar eu nu mă tem, dar eu nu mă tem!
  \end{verse}

  \begin{verse}
    Când oamenii cei răi mă-nconjoară cu ură, \\
    Din cupa lor să-mi dea să beau o picătură, \\
    Dar eu nu mă tem, dar eu nu mă tem.
  \end{verse}

  \begin{verse}
    Cu Domnul meu eu trec biruitor peste toate, \\
    Din apă și din foc să mă scape El poate, \\
    Dar eu nu mă tem, dar eu nu mă tem!
  \end{verse}

  \end{song}
\end{document}