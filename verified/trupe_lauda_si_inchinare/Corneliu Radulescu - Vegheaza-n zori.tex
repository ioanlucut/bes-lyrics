% =====================================================================
% This file is auto-generated.
%
% Copyright (c) 2024 Ioan Lucuț (ioan.lucut88@gmail.com)
%
% Do not edit this file directly. Any changes made will be overwritten
% the next time the file is generated.
% =====================================================================

% This is the only preamble definition required
\documentclass{leadsheet}
\begin{document}

  %  https://tex.stackexchange.com/questions/9852/what-is-the-difference-between-page-break-and-new-page
  %  Every song should start in a new page.
  \newpage

  \begin{song}{
     title={Veghează-n zori},
     subtitle={v1,c,v2,c,v3,c},
     composer={Corneliu Radulescu}
  }

  \begin{verse}
    ^{D}Veghează-n zori, când ziua ^{G}iar zâm^{D}bește, \\
    Când totul ^{A7}este iar învio^{D}rat, \\
    În tine poate încă ^*{G}dăinu ^{D}iește \\
    Vreo umbră ^{A7}neagră, plină de pă^{D}cat.
  \end{verse}

  \begin{chorus}
    ^{G}Iar în amurg, când ^{D}totul se-odihne^{Em}ște, \\
    Tu-ndreaptă-ți ^{A}gândul ^{G-A}iarăși spre I^{D}sus. \\
    Îi cântă ^{G}laudă ^{D}și Îi mulțu^{Em}mește: \\
    Pe a Lui că^{D-A}rare Domnul ^{A7}iarăși te-a ^{D}adus.
  \end{chorus}

  \begin{verse}
    Veghează-n orice clipă-a vieții tale, \\
    Chiar și atunci când totul e senin, \\
    Căci firea ta e rea din cale-afară \\
    Și-n jurul tău e un dușman hain.
  \end{verse}

  \begin{verse}
    Și la amiază-n munca grea a zilei, \\
    Și în vâltoarea ei tu să veghezi. \\
    În rugăciunea dragostei și-a milei, \\
    Doar cu Isus rămâi și să lucrezi.
  \end{verse}

  \end{song}
\end{document}