% =====================================================================
% This file is auto-generated.
%
% Copyright (c) 2024 Ioan Lucuț (ioan.lucut88@gmail.com)
%
% Do not edit this file directly. Any changes made will be overwritten
% the next time the file is generated.
% =====================================================================

% This is the only preamble definition required
\documentclass{leadsheet}
\begin{document}

  %  https://tex.stackexchange.com/questions/9852/what-is-the-difference-between-page-break-and-new-page
  %  Every song should start in a new page.
  \newpage

  \begin{song}{
     title={Pentru noi e taină mare},
     subtitle={v1.1,v1.2,c,v2.1,v2.2,c,v3.1,v3.2,c},
     composer={Speranța și prietenii}
  }

  \begin{verse}
    Pentru ^{A}noi e taină mare ce va ^{D}fi în vii^{A}tor, \\
    Poate ^{D}e o zi cu ^{A}soare, poate ^{F#m}e o zi cu ^{E-G#}nor.
  \end{verse}

  \begin{verse}
    Însă ^{A}știu că-ngrijorarea să o ^{D}las va tre^{A}bui, \\
    Dumne^{D}zeu și îndu^{A}rarea în ^{F#m}etern ^{E7}vor dăi^{A}nui.
  \end{verse}

  \begin{chorus}
    /: Nu pot ^{D}ști ce fi-va ^{A}mâine, însă ^{B7}știu că Dumne^{E}zeu \\
    Îngri^{A}jește și de ^{D}mine căci și ^{A-E}eu sunt ^{E7}fiul ^{A}Său. :/
  \end{chorus}

  \begin{verse}
    Parcă urc trepte de aur, către culmi de har ceresc \\
    Și zăresc măreț tezaur, oare când îl dobândesc?
  \end{verse}

  \begin{verse}
    E un loc de strălucire, curcubeu multicolor. \\
    Când voi fi în nemurire cu al meu Mântuitor?
  \end{verse}

  \begin{verse}
    Ce va fi e taină mare, s-ar putea s-ajung sărac, \\
    S-ar putea să n-am mâncare, sau să n-am ce să îmbrac.
  \end{verse}

  \begin{verse}
    Pentru flori și păsărele, Dumnezeu e Tată bun, \\
    Însă vrea, prin orice rele, eul meu să i-L supun.
  \end{verse}

  \end{song}
\end{document}