% =====================================================================
% This file is auto-generated.
%
% Copyright (c) 2024 Ioan Lucuț (ioan.lucut88@gmail.com)
%
% Do not edit this file directly. Any changes made will be overwritten
% the next time the file is generated.
% =====================================================================

% This is the only preamble definition required
\documentclass{leadsheet}
\begin{document}

  %  https://tex.stackexchange.com/questions/9852/what-is-the-difference-between-page-break-and-new-page
  %  Every song should start in a new page.
  \newpage

  \begin{song}{
     title={Rabuni, Tu din morți ai înviat!},
     subtitle={c,v1,c,v2,c,v3,c},
     composer={Romeo Pelle și Dănut Noane}
  }

  \begin{chorus}
    /: Ra^{G}buni, Tu din ^{D}morți ai în^{Em}viat, \\
    Fii în ^{C}veci ^*{D}rifi ^{G}cat! ^{C2-D7} \\
    Ra^{G}buni, Tu din ^{D}morți ai în^{Em}viat, \\
    Cu Cel ^{C}Sfânt ne-ai ^*{D}împă ^{G}cat. ^{C2-D7}:/
  \end{chorus}

  \begin{verse}
    ^{Am}Ce fior adânc în ^{Em}suflet ne-a pătruns \\
    A^{C}tunci când ^{D}pe Golgota ^{C2}ai mu^{G}rit! \\
    Cre^{Am}deam că Te-am pierdut, pe cru^{Em}ce și-n mormânt, \\
    Dar ^{C}viață ^{D}ești și veșnic neîn^{B4-B}frânt. ^{D-C}
  \end{verse}

  \begin{verse}
    Lumin-a răsărit în sufletu-mi trudit, \\
    Căci lepădat am fost și-ndepărtat. \\
    Nădejdea mi-ai redat, o, cât de minunat! \\
    Din morți am înviat cu Tine-odat’.
  \end{verse}

  \begin{verse}
    Cum aș putea să tac, cum aș putea să fac \\
    Ca să nu spun despre iubirea Ta? \\
    O lume-ntreagă aș vrea să creadă-n jertfa Ta \\
    Și să-i îmbraci cu slava Ta.
  \end{verse}

  \end{song}
\end{document}