% =====================================================================
% This file is auto-generated.
%
% Copyright (c) 2024 Ioan Lucuț (ioan.lucut88@gmail.com)
%
% Do not edit this file directly. Any changes made will be overwritten
% the next time the file is generated.
% =====================================================================

% This is the only preamble definition required
\documentclass{leadsheet}
\begin{document}

  %  https://tex.stackexchange.com/questions/9852/what-is-the-difference-between-page-break-and-new-page
  %  Every song should start in a new page.
  \newpage

  \begin{song}{
     title={Mai mult acum ca niciodată},
     subtitle={v1.1,v1.2,c,v2.1,v2.2,c,v3.1,v3.2,c},
     composer={Puiu Chibici}
  }

  \begin{verse}
    ^{G}Mai mult acum, ca ^*{C}nicio ^{G}dată, \\
    E drumul ^{D7}mai alune^{G}cos. \\
    Credința-i parcă ^{C}pmai ui^{G}tată, \\
    Tot mai în ^{D7}umbră-i pus Hris^{G}tos.
  \end{verse}

  \begin{verse}
    ^{B}E timpul azi pentru trezire, ^{Em} \\
    E timpu-a^{A7}proape spre apus, ^{D7} \\
    Acum în^{G}treaga omeni^{C}re \\
    Are ne^{G}voie ^{D7}de I^{G}sus.
  \end{verse}

  \begin{chorus}
    /: ^{C}Mai mult acum (mai mult acum), \\
    Ca nicio^{G}dată (mai mult acum), \\
    E vremea ^{D7}să ne ridi^{G}căm \\
    Să Îl che^{C}măm, ca altă^{G}dată, \\
    Și îndea^{D7}proape să Îl ur^{G}măm. :/
  \end{chorus}

  \begin{verse}
    E timpul pentru apropiere, \\
    Pentru căință-n rugăciuni, \\
    Isus mai scrie grațiere, \\
    Deci să-ncercăm să fim mai buni.
  \end{verse}

  \begin{verse}
    Să ne oprim din alergare, \\
    Să stăm o clipă să privim \\
    La ce se-ntâmplă în lumea mare \\
    Și sincer să ne pocăim.
  \end{verse}

  \begin{verse}
    Mai face Domnul iar chemare, \\
    Poate e ultimul apel, \\
    Fă-ți azi și tu o cercetare \\
    Și-ntoarce-te acum la El.
  \end{verse}

  \begin{verse}
    Tot mai aproape e sfârșitul, \\
    Curând, curând El va veni. \\
    Prinde, prietene, momentul \\
    Și fericit pe veci poți fi.
  \end{verse}

  \end{song}
\end{document}