% =====================================================================
% This file is auto-generated.
%
% Copyright (c) 2024 Ioan Lucuț (ioan.lucut88@gmail.com)
%
% Do not edit this file directly. Any changes made will be overwritten
% the next time the file is generated.
% =====================================================================

% This is the only preamble definition required
\documentclass{leadsheet}
\begin{document}

  %  https://tex.stackexchange.com/questions/9852/what-is-the-difference-between-page-break-and-new-page
  %  Every song should start in a new page.
  \newpage

  \begin{song}{
     title={Mai mult decât așteaptă străjerul dimineața},
     subtitle={v1.1,v1.2,v2.1,v2.2,v3.1,v3.2,v4.1,v4.2},
     composer={Nicolae Moldoveanu},
     lyrics={Nicolae Moldoveanu}
  }

  \begin{verse}
    ^{d}Mai mult de^{g}cât așteaptă \\
    ^{A}străje^{A7}rul di^{d}mineața, \\
    ^{d}Te-așteaptă-al nos^{g}tru su^{A}flet, \\
    Isus, să-Ți ^{A7}va^{D}dă fața.
  \end{verse}

  \begin{verse}
    /: ^{d}Cum soare^{g}le răsare \\
    ^{C}Și luminea^{F}ză-ntinsul, \\
    A^{D}șa ne um^{g}pli via^{A}ța, \\
    Isus, Tu, ^{A7}necu^{d}prinsul. :/
  \end{verse}

  \begin{verse}
    Mai mult decât așteaptă \\
    Un mugur primăvara, \\
    Mai mult decât așteaptă \\
    Înstrăinatul țara,
  \end{verse}

  \begin{verse}
    /: Mai mult decât așteaptă \\
    Ogorul ploaia lină, \\
    Dorim, Isuse, fața \\
    Și slava Ta să vină. :/
  \end{verse}

  \begin{verse}
    Mai mult decât așteaptă \\
    Bolnavul vindecarea, \\
    Noi așteptăm, Isuse, \\
    În țara Ta intrarea.
  \end{verse}

  \begin{verse}
    /: Mai mult decât așteaptă \\
    Logodnica pe mire, \\
    Dorim întâmpinarea \\
    Venirii în mărire. :/
  \end{verse}

  \begin{verse}
    Mai mult decât așteaptă \\
    Robiții ‘liberarea, \\
    Corăbierii portul, \\
    Trudiții înserarea,
  \end{verse}

  \begin{verse}
    /: Ostașii biruința \\
    Și-n lupta lor cununa, \\
    Noi Te-așteptăm, Isuse, \\
    Pe Tine-ntotdeauna. :/
  \end{verse}

  \end{song}
\end{document}