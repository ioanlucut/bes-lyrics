% =====================================================================
% This file is auto-generated.
%
% Copyright (c) 2024 Ioan Lucuț (ioan.lucut88@gmail.com)
%
% Do not edit this file directly. Any changes made will be overwritten
% the next time the file is generated.
% =====================================================================

% This is the only preamble definition required
\documentclass{leadsheet}
\begin{document}

  %  https://tex.stackexchange.com/questions/9852/what-is-the-difference-between-page-break-and-new-page
  %  Every song should start in a new page.
  \newpage

  \begin{song}{
     title={Mergem înainte!},
     subtitle={c,v1,c,v2,c,v3,c,v4,c},
     composer={Beni Cibu}
  }

  \begin{verse}
    ^{G}Tu-ai fost un loc de adă^{D}post, \\
    Al ^{C}nostru, Doamne, din veci^{G}e \\
    Și sub al aripii um^{D}brar \\
    Noi am găsit credincio^{G}șie. \\
    Ce mi^{Em}nunat, ^{C}ce minu^{G}nat, \\
    Să ai o Stâncă-n vremuri ^{D}grele, \\
    Ce minunat, ce minu^{G}nat!
  \end{verse}

  \begin{chorus}
    ^{C}Mergem înainte cu Tine, ^{G}Doamne, \\
    Orice ar ve^{D}ni, orice s-ar întâm^{G}pla! \\
    ^{C}Mergem înainte cu Tine, ^{G}Doamne, \\
    Nu ne clăti^{D}năm, suntem în mâna ^{G}Ta!
  \end{chorus}

  \begin{verse}
    S-a ridicat al mării val \\
    Peste corabie deodată \\
    Și când credeam că norii grei \\
    Îmi vor umbri viața toată \\
    M-ai întărit, m-ai întărit. \\
    Ca să rămân pe Stânca tare, \\
    M-ai întărit, m-ai întărit!
  \end{verse}

  \begin{verse}
    Cu Tine am rămas mereu \\
    Și voi rămâne totdeauna \\
    Căci mare este harul Tău \\
    Ce l-am cântat și-l cânt întruna. \\
    Te voi slăvi, Te voi slăvi, \\
    Pe Tine, dorul dorurilor mele, \\
    Te voi slăvi, Te voi slăvi!
  \end{verse}

  \begin{verse}
    Ce va fi mâine eu nu știu, \\
    Dar știu că taina voii Tale \\
    O voi pricepe în curând, \\
    Când voi cânta-n cer osanale! \\
    Ce minunat, ce minunat, \\
    Când voi ajunge-n strălucire, \\
    Ce minunat, ce minunat!
  \end{verse}

  \end{song}
\end{document}