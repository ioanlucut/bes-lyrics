% =====================================================================
% This file is auto-generated.
%
% Copyright (c) 2024 Ioan Lucuț (ioan.lucut88@gmail.com)
%
% Do not edit this file directly. Any changes made will be overwritten
% the next time the file is generated.
% =====================================================================

% This is the only preamble definition required
\documentclass{leadsheet}
\begin{document}

  %  https://tex.stackexchange.com/questions/9852/what-is-the-difference-between-page-break-and-new-page
  %  Every song should start in a new page.
  \newpage

  \begin{song}{
     title={Ce dar nemeritat},
     subtitle={v1,c,v2,c,b,c},
     composer={Philip Wickham; Josh Farro & Jeremy Riddle},
     lyrics={Philip Wickham; Josh Farro & Jeremy Riddle; Florin Mesaroș & BBSO}
  }

  \begin{verse}
    ^{D}Cine transformă în soare veșnic \\
    ^{D-G}Orice fărâmă de întuneric? \\
    ^{A4-B}E Domnul gloriei, ^{A}e Domnul ^*{D}dom ^*{G}ni ^{A}lor \\
    ^{D}Cine vorbește și-a Lui vorbire \\
    ^{D-G}Zguduie munții din temelie? \\
    ^{A4-B}E Domnul gloriei, ^{A}e Domnul domni^{D}lor! ^{G2 A}
  \end{verse}

  \begin{chorus}
    ^{N.C}Ce dar nemeri^{D}tat să pot să fiu sal^{D-G}vat! \\
    ^{D}În ceasul cel mai ^{D-B}greu ai luat păcatul ^{A4A}meu, \\
    ^{D}Moartea ai biruit ca eu să fiu sfin^{G2}țit. \\
    ^{Bm}Astăzi cânt liber, ^{A}Isus m-a mântu^{D}it! ^{(G2/A)}
  \end{chorus}

  \begin{verse}
    Cine mai poate azi să ridice \\
    Pe cei căzuți chiar ca fii și fiice? \\
    E Domnul gloriei, e Domnul domnilor! \\
    Cine aduce la El popoare, \\
    Plin de dreptate și îndurare? \\
    E Domnul gloriei, e Domnul domnilor!
  \end{verse}

  \begin{bridge}
    /: ^{D}Vrednic este Mielul ‘junghiat, \\
    ^{G2}Vrednic e Împăratul din morți înviat! :/ (3x) \\
    ^{D-B}Vrednic este Mielul ‘junghiat, \\
    ^{D-G}Vrednic, vrednic, vred^{G}nic!
  \end{bridge}

  \end{song}
\end{document}