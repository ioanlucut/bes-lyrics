% =====================================================================
% This file is auto-generated.
%
% Copyright (c) 2024 Ioan Lucuț (ioan.lucut88@gmail.com)
%
% Do not edit this file directly. Any changes made will be overwritten
% the next time the file is generated.
% =====================================================================

% This is the only preamble definition required
\documentclass{leadsheet}
\begin{document}

  %  https://tex.stackexchange.com/questions/9852/what-is-the-difference-between-page-break-and-new-page
  %  Every song should start in a new page.
  \newpage

  \begin{song}{
     title={La Golgota demult Isus},
     subtitle={v1,c1.1,c1.2,v2,c1.1,c1.2,v3,c1.1,c1.2,b,c2.1,c2.2},
     composer={Dan Damian}
  }

  \begin{verse}
    ^{F}La Golgota de^{A}mult Isus \\
    Prin sânge ^{D}sfânt, plăti de-a^{Gm}juns \\
    Darul di^{C}vin al mântu^{F}irii prin ^{Bb}Hristos \\
    Har pentru ^*{C}o ^{C7}mul mort, păcă^{F}tos.
  \end{verse}

  \begin{chorus}
    ^{F}Biserica Lui ^{D}biruitoare va fi ^{Gm}mereu pân’ la sfârșit, \\
    ^{Gm}Nimeni nu ^{C}poate să o învingă, El a pro^{F}mis! ^{C}
  \end{chorus}

  \begin{chorus}
    Nu e pu^{F}tere, nu sunt sis^{E}teme, nu-s legi fă^{Gm}cute pe pământ \\
    Să ne do^{G}boare, să ne oprească, Cu noi e Domnul ^{F}sfânt!
  \end{chorus}

  \begin{verse}
    A înviat, S-a înălțat, la Cincizecimi a revărsat \\
    Duhul Cel Sfânt, biserica Lui botezând, \\
    Plin de putere cu El umblând.
  \end{verse}

  \begin{verse}
    Trecut-au regi și-mpărății, stăpâni și vremi cu oameni mii, \\
    Biserica triumfătoare a rămas, cu ea e Domnul, pas cu pas!
  \end{verse}

  \begin{chorus}
    Isus revine, pe nori ca Mire să-și ia la cer Biserica \\
    Sfântă, curată, fără de pată, El a promis!
  \end{chorus}

  \begin{chorus}
    Ce întâlnire, în strălucire, când fața Lui o vom vedea, \\
    Cu adorare, în închinare, Isus Te-om lăuda!
  \end{chorus}

  \begin{bridge}
    /: Glorie ^{F}Tatălui, glorie Fi^{F}ului, \\
    Glorie ^{C}Duhului Cel S^{C7}fânt, a^{F}min! :/
  \end{bridge}

  \end{song}
\end{document}