% =====================================================================
% This file is auto-generated.
%
% Copyright (c) 2024 Ioan Lucuț (ioan.lucut88@gmail.com)
%
% Do not edit this file directly. Any changes made will be overwritten
% the next time the file is generated.
% =====================================================================

% This is the only preamble definition required
\documentclass{leadsheet}
\begin{document}

  %  https://tex.stackexchange.com/questions/9852/what-is-the-difference-between-page-break-and-new-page
  %  Every song should start in a new page.
  \newpage

  \begin{song}{
     title={Oare cum va fi în slavă?},
     subtitle={v1.1,v1.2,c,v2.1,v2.2,c,v3.1,v3.2,c},
     composer={Damaris Căuneac}
  }

  \begin{verse}
    Oare ^{D}cum va ^{A}fi în ^{Bm}slavă, pe-^{D}nălți^{A}mea ce^{G2}rului \\
    Când vom ^{D}sta fa^{A}ță-n ^{Bm}față, pri^{D}vind sla{A}va Mie^{G2}lului?
  \end{verse}

  \begin{verse}
    Da, știm, ^{D}ziua ^{A}nu-i de^{Bm}parte ^{D}și Isus ^{A}va ^{G2}reveni, \\
    El, nă^{D}dejdea ^{A}noa^{Bm}stră vie, în ^{D}curând ^{A}cu El ^{G2}vom fi.
  \end{verse}

  \begin{chorus}
    Vom ve^{G}dea Ieru^{A}sali^{D}mul dorit \\
    Ce ne-a ^{G}fost ^{A}de Tatăl ^{D}pre^{A}gătit. \\
    Vom ^{G}înălța ^{A}cân^{Bm}ta^{D}rea, ^{A}imnul prea^{D}slăvit, \\
    Cu ^{G}El pururi vom ^{Bm}rămâne \\
    Dinco^{Em}lo de zări, ^{A}pe culme, în ^{D///}veșnicii.
  \end{chorus}

  \begin{verse}
    Oare cum va fi în slavă, pe Isus Îl vom privi? \\
    Vom putea rosti cuvinte sau tăcerea va domni?
  \end{verse}

  \begin{verse}
    Oare cum va fi cântarea în preasfinte armonii, \\
    Vom putea rosti cuvinte sau tăcerea va domni?
  \end{verse}

  \begin{verse}
    Oare cum va fi cântarea celor sfinți ce-au biruit \\
    Și în viața lor, pe Isus, cu iubire L-au slujit?
  \end{verse}

  \begin{verse}
    Cum va fi Ierusalimul în lumină strălucind? \\
    Razele iubirii Sale, nu ne-om sătura privind.
  \end{verse}

  \end{song}
\end{document}