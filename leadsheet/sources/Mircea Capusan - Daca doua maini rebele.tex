% =====================================================================
% This file is auto-generated.
%
% Copyright (c) 2024 Ioan Lucuț (ioan.lucut88@gmail.com)
%
% Do not edit this file directly. Any changes made will be overwritten
% the next time the file is generated.
% =====================================================================

% This is the only preamble definition required
\documentclass{leadsheet}
\begin{document}

  %  https://tex.stackexchange.com/questions/9852/what-is-the-difference-between-page-break-and-new-page
  %  Every song should start in a new page.
  \newpage

  \begin{song}{
     title={Dacă două mâini rebele},
     subtitle={v1,v2,v3,v4},
     composer={Mircea Capusan},
     lyrics={Costache Ioanid}
  }

  \begin{verse}
    Dacă ^{E}două mâini rebele nu te-ar ^{A}ține orb, mereu, \\
    Te-ai o^{B}pri în clipe grele, și, pri^{A}vind cu dor la ste^{E}le, \\
    /: Ai ve^{E-C#}dea că-i scris pe ^{B7}ele: „^{E}Dum^{F#m}ne^{E-E}zeu”! ^{E} ^{A-E} ^{E}:/
  \end{verse}

  \begin{verse}
    Dacă două vechi zăvoare nu te-ar ține surd mereu, \\
    Ai șopti: „Ce meșter oare scoate păsări cântătoare?” \\
    /: Și ți-ar spune orice floare: „Dumnezeu”! :/
  \end{verse}

  \begin{verse}
    Dacă marea întristării nu te-ar ține rob mereu, \\
    Ai striga în largul zării: „Cine-i Tatăl îndurării?” \\
    /: Și ți-ar spune valul mării: „Dumnezeu”! :/
  \end{verse}

  \begin{verse}
    Dacă ți-ai vedea veșmântul care-mbracă vechiul eu, \\
    Ai uda de plâns pământul și primind cu drag Cuvântul, \\
    /: L-ai simți în piept pe sfântul Dumnezeu! :/
  \end{verse}

  \end{song}
\end{document}